% ============================== BIT ALGORITHMS ============================== %
% Project:          The Experimental Bit Algorithms Library
% Name:             bit-algorithms.tex
% Description:      Scientific and technical paper on bit algorithms
% Creator:          Vincent Reverdy
% Contributor(s):   Vincent Reverdy [2019]
% License:          BSD 3-Clause License
% ============================================================================ %



% =============================== CONFIGURATION ============================== %
% Document class
\documentclass[letterpaper, 8pt, twocolumn]{article}
% ---------------------------------------------------------------------------- %
% Packages
\usepackage[
    tmargin = 1in,
    bmargin = 1in,
    lmargin = 1in,
    rmargin = 1in
]{geometry}
\usepackage{lipsum}
\usepackage{lmodern}
\usepackage{titling}
\usepackage{titlesec}
\usepackage[sf=false]{libertine}
\usepackage{cite} \usepackage{array}
\usepackage{amsmath}
\usepackage{amssymb}
\usepackage{amsthm}
\usepackage{mathtools}
\usepackage{algpseudocode}
\usepackage{algorithm}
\usepackage{float}
\usepackage{fdsymbol}
\usepackage{tabularx}
\usepackage{epigraph}
\usepackage{graphicx}
\usepackage{fancyhdr}
\usepackage{siunitx}
\usepackage{colortbl}
\usepackage{enumitem}
\usepackage[listings,skins,theorems,breakable,most]{tcolorbox}
\edef\restoreparindent{\parindent=\the\parindent\relax}
\usepackage[parfill]{parskip}
\restoreparindent
\usepackage{hyperref}
% ---------------------------------------------------------------------------- %
% Compilation
\pdfpageattr{/Group << /S /Transparency /I true /CS /DeviceRGB>>}
% ---------------------------------------------------------------------------- %
% Bibliography
\bibliographystyle{apalike}
% ============================================================================ %



% ================================== LAYOUT ================================== %
% Page style
\pagestyle{fancy}
\fancyhead{}
\renewcommand{\headrulewidth}{0pt}
\renewcommand{\footrulewidth}{0pt}
% ---------------------------------------------------------------------------- %
% Fonts
\renewcommand{\ttdefault}{cmtt}
% ---------------------------------------------------------------------------- %
% Spacing
\titlespacing\section{0pt}{0.25\baselineskip plus 2pt minus 1pt}{0.125\baselineskip plus 2pt minus 1pt}
\titlespacing\subsection{0pt}{0.25\baselineskip plus 2pt minus 1pt}{0.125\baselineskip plus 2pt minus 1pt}
\titlespacing\subsubsection{0pt}{0.25\baselineskip plus 2pt minus 1pt}{0.125\baselineskip plus 2pt minus 1pt}
\setlist{topsep=-0.25\baselineskip, itemsep=0.25\baselineskip, partopsep=0.\baselineskip, parsep=0.\baselineskip}
% ============================================================================ %



% ================================= COMMANDS ================================= %
% Acronyms
\newcommand{\acronym}[1]{\textsc{#1}}
\newcommand{\acrolang}[1]{\textsc{#1}}
\newcommand{\cpp}{\acrolang{C++}}
\newcommand{\clang}{\acrolang{C}}
\newcommand{\typecell}[3]{\cellcolor{#1}\textcolor{blue}{#2}#3}

\DeclarePairedDelimiter\ceil{\lceil}{\rceil}
\DeclarePairedDelimiter\floor{\lfloor}{\rfloor}
% ============================================================================ %



% ================================ INFORMATION =============================== %
% Information
\title{Towards a Standard Library of Bit Manipulation Algorithms}
\author{%
Vincent Reverdy%
%\and FirstName LastName%
}
\date{\today}
% ---------------------------------------------------------------------------- %
% Hyperref
\hypersetup{
    pdfauthor={\theauthor},
    pdftitle={\thetitle},
    pdfsubject={Bit manipulation algorithms},
    pdfkeywords={high performance computing, programming languages, algorithms, bit, bit manipulation, C++},
    pdfproducer={LaTeX},
    pdfcreator={pdflatex},
    colorlinks,
    citecolor=blue,
    filecolor=blue,
    linkcolor=blue,
    urlcolor=blue
}
% ============================================================================ %



% =================================== MAIN =================================== %
\begin{document}
\maketitle
% ---------------------------------------------------------------------------- %
\begin{abstract}
\label{sec:abstract}
\end{abstract}
% ---------------------------------------------------------------------------- %
\section{Introduction}
\label{sec:introduction}
% ---------------------------------------------------------------------------- %
\section{Algorithms}
\label{sec:algorithms}

\paragraph{Notation}


\subsection{Non-modifying Algorithms}
\label{subsec:nonmodifyingAlgs}
\subsubsection{\texttt{max\_element}}
\paragraph{Description}
Locates the largest element in a range \texttt{[first, last)} and returns an iterator to the element.

\paragraph{Implementation}
Implementation is somewhat trivial. If \texttt{first} is LSB aligned (meaning it points to the 0th bit of a word) we perform a partial read from \text{first} to the closer of the MSB and the \text{last} iterator. Missing bits from the read are padded with 0 bits in the versions of the algorithm where a custom comparator is not passed. Next, we check the read word to see if it is greater than zero. If it is greater than zero, we know there is a set bit somewhere within the word. We can find the exact position of the set bit using \texttt{\_tzcnt} and return this position as an iterator. If \texttt{first} is not equal to \texttt{last}, we advance first to the 0th bit of the next word.

While \texttt{first} is not equal to \texttt{last} and \texttt{last} is not in the same word as \texttt{first} (marking a final and partial word), we read whole words as normal and check for set bits as previously described.

If we arrive at a final, partial word, we again take a partial read of the bits remaining in the range, padding any missing bits with 0. We check for set bits as normal and return an iterator if a set bit was found.

If we have scanned the entire range and no set bits were found, we simply return \texttt{first}.

\subsection{Modifying Algorithms}
\label{subsec:modifyingAlgs}

\subsubsection{\texttt{copy}}
\label{subsubsec:copy}
\paragraph{Description}
Copies all elements in the range [first, last) starting from first and 
proceeding to last - 1. The behavior is undefined if d\_first is within the 
range [first, last). In this case, std::copy\_backward may be used instead.

\paragraph{Implementation}
See psuedocode below:
\begin{algorithm}[H]
    \caption{Current copy implementation}
    \begin{algorithmic}[1]
        \Function{copy}{\texttt{first1, last1, first2}}
        \If {!\texttt{is\_aligned}(\texttt{first2})}
            \State Use \texttt{get\_word} to copy bits and align \texttt{first2}
            \State Advance \texttt{first1} correspondingly.
        \EndIf
        \While {There are remaining bits to copy} 
            \State Construct words of \texttt{dst\_word\_type}
            from the source and assign to destination.
        \EndWhile
        \State If the last iterator is not aligned, copy the remaining bits to the
            final word.
        \EndFunction
    \end{algorithmic}
\end{algorithm}
\paragraph{Comments}
\begin{itemize}
    \item Might be worth thinking about having separate implementations 
        depending on the size of the words using \texttt{if constexpr}.
\end{itemize}

\subsubsection{\texttt{shift\_left} and \texttt{shift\_right}}
\label{subsubsec:shift}
\paragraph{Description}
\begin{itemize}
    \item[\texttt{shift\_left}] Shifts the elements towards the beginning of the 
            range. If $n \leq 0 || n \geq last - first$, there are no effects. 
            Otherwise, for every integer $i$ in $[0, last - first - n)$, moves the 
            element originally at position $first + n + i$ to position $first + i$. 
            The moves are performed in increasing order of i starting from.
    \item[\texttt{shift\_right}] Shifts the elements towards the end of the 
        range. If $n \leq 0 || n \geq last - first$, there are no effects. Otherwise, 
        for every integer $i$ in $[0, last - first - n)$, moves the element 
        originally at position $first + i$ to position $first + n + i$. If ForwardIt 
        meets the LegacyBidirectionalIterator requirements, then the moves are 
        performed in decreasing order of $i$ starting from $last - first - n - 1$.
\end{itemize}

\paragraph{Implementation}
For both left and right shift, let $N$ be the number of total word shifts and 
$n'$ be the remainder of needed bit shifts i.e.
$$N=\floor*{\frac{n}{|\texttt{word\_type}|}}$$ $$n'=n-N*|\texttt{word\_type}|$$
\subparagraph{\texttt{shift\_left}}

\begin{algorithm}[H]
    \caption{Current \texttt{shift\_left} implementation}
    \begin{algorithmic}[1]
        \Function {shift\_left}{\texttt{first, last, n}}
            \State Save first and last words if not aligned.
            \State Unset bits in first word that are not part of 
            the range.
            \State Use STL version to shift words by $N$.
            \State Create latent iterator to keep track of penultimate word.
            \State \texttt{\_shrd} each word and the next by $n'$ until we 
            reach the last word
            \State Right shift the last word by $n'$.
            \State Put back the out of range bits if necessary.
        \EndFunction
    \end{algorithmic}
\end{algorithm}

\begin{algorithm}[H]
    \caption{Current \texttt{shift\_right} implementation}
    \begin{algorithmic}[1]
        \Function {shift\_left}{\texttt{first, last, n}}
            \State Save first and last words if not aligned.
            \State Unset bits in last word that are not part of 
            the range.
            \State Use STL version to shift words by $N$.
            \State Create two temporary variables \texttt{T1} and \texttt{T2}
            \State \texttt{T1 = *first.base()}
            \State \texttt{*first.base() <<= n'}
            \For {\texttt{it=first.base(), it <= last, it++}}
                \State \texttt{T2 = *it}
                \State \texttt{*it = \_shld(*it, T1, n')}
                \State \texttt{T1 = T2}
            \EndFor
            \State Put back the out of range bits if necessary.
        \EndFunction
    \end{algorithmic}
\end{algorithm}

\paragraph{Comments}
\begin{itemize}
    \item \texttt{shift\_right} even with a ForwardIt may be able to avoid
        having \textbf{two} temporary variables.
    \item If we have a bidiriectional iterator, \texttt{shift\_right} 
        can avoid temporary variables and just use \texttt{\_shld} starting 
        from last and going to first.
\end{itemize}
% ---------------------------------------------------------------------------- %
\bibliography{bit-algorithms}
% ---------------------------------------------------------------------------- %
\end{document}
% ============================================================================ %

